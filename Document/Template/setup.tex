\documentclass[aspectratio=1610,10pt]{beamer}
\usepackage[T1]{fontenc}
\usepackage[utf8]{vietnam}
\usepackage{amsmath}
\usefonttheme[onlymath]{serif}
\usepackage{tikz}
\usepackage[most]{tcolorbox}
\usepackage{listingsutf8}
\usepackage{comment}
\usepackage{tabularx}
\usepackage{mathtools}
\usepackage{algpseudocode}
\usepackage{xcolor}
\setbeamertemplate{caption}[numbered]
\usepackage{makecell}
\usepackage{fnpct}
\usepackage{multirow}
\usepackage{bbm}
\usepackage{pgfpages}
\usepackage{handoutWithNotes}
\usepackage{xcolor}
\usepackage{calc}
\usepackage{changepage}




% Bỏ thanh mờ mờ góc phải
\setbeamertemplate{navigation symbols}{}


\setbeamercolor{section in head/foot}{bg=cyan!5, fg=cyan!50}
\setbeamercolor{author in head/foot}{bg=cyan!5, fg=black}

\setbeamerfont{section in head/foot}{size=\small}
\setbeamerfont{author in head/foot}{size=\small}

% \setbeamertemplate{footline}{
%   \begin{beamercolorbox}[ht=3ex,dp=1.5ex,right]{author in head/foot}
%     \insertframenumber{} / \inserttotalframenumber\hspace*{2ex}
%   \end{beamercolorbox}
% }

% \setbeamertemplate{headline}{%
%   \begin{beamercolorbox}[wd=\paperwidth,ht=5ex,dp=2.5ex]{section in head/foot}
%     \hspace{5 ex}
%      \insertsectionhead
%     \ifx\insertsubsectionhead\empty
%       % Không có subsection, không hiển thị dấu gạch ngang
%     \else
%       % Có subsection, hiển thị dấu gạch ngang và tên của subsection
%       \,---\,\insertsubsectionhead
%     \fi
%   \end{beamercolorbox}%
%   \vspace{5 pt}
% }

\setbeamertemplate{footline}{
    \begin{beamercolorbox}[ht=5ex,dp=3ex,right]{author in head/foot} % Tăng kích thước box
        \usebeamerfont{author in head/foot}\insertframenumber{} / \inserttotalframenumber\hspace*{2ex}
    \end{beamercolorbox}
}

\setbeamertemplate{headline}{
    \begin{beamercolorbox}[wd=\paperwidth,ht=6ex,dp=3.5ex]{section in head/foot} % Tăng kích thước box
        \usebeamerfont{section in head/foot}\hspace{1 ex}
        \insertsectionhead
        \ifx\insertsubsectionhead\empty
            % Không có subsection, không hiển thị dấu gạch ngang
        \else
            % Có subsection, hiển thị dấu gạch ngang và tên của subsection
            \,---\,\insertsubsectionhead
        \fi
    \end{beamercolorbox}%
}


\everymath{\displaystyle}


\setbeamerfont{title}{size=\fontsize{26}{12}}

% Tạo màu cho các block
\setbeamercolor{block title example}{fg=white, bg=teal!80!}
\setbeamercolor{block body example}{ bg=teal!20}
\setbeamercolor{block title alerted}{fg=white, bg=orange}
\setbeamercolor{block body alerted}{ bg=orange!25}
\setbeamercolor{block title}{bg=cyan, fg=white}
\setbeamercolor{block body}{bg=cyan!10}

% Dấu tròn trước item
\setbeamertemplate{itemize item}{\textbullet}


% Tên nè
\title{Tiny Shell}
\subtitle{
    \Large Tương tác với hệ điều hành thông qua giao diện dòng lệnh\\
    \vspace{5 pt}\small\textbf{Học phần:} IT3070 - Nguyên lý Hệ điều hành\\ \small\textbf{Giảng viên hướng dẫn:} TS. Phạm Đăng Hải}
\author{N.~V.~T.~Kiệt \inst{1,2}, B.~Q~Phong \inst{1,2}, L.~T.~Khang \inst{1}, N.~T.Tuyển \inst{1}}


\institute %
{
    \inst{1}%
    Chương trình tài năng - Khoa học máy tính K67\\
    Trường Công nghệ Thông tin và Truyền thông
    \and
    \inst{2}%
    Phòng thí nghiệm Mô hình hóa, Mô phỏng và Tối ưu hóa\\
    Trung tâm nghiên cứu quốc tế về Trí tuệ nhân tạo, BKAI

}
\date{\today}

% \setbeamertemplate{frametitle}{
% % Bọc tcolorbox bằng hbox để giới hạn chiều rộng theo nội dung
%       \begin{tcolorbox}[
%         colback=cyan!10,
%         colframe=white, % Màu viền của hộp
%         boxrule=0mm, % Độ dày của viền, đặt là 0 nếu không muốn viền
%         arc=3mm, % Độ cong của góc Tự động làm cho các góc ngoài cong theo arc được đặt
%         boxsep=1mm, % Khoảng cách giữa viền và nội dung trong hộp
%         left=3mm, right=3mm, top=1mm, bottom=1mm % Padding bên trong hộp
%       ]
%         \insertframetitle % Chèn tiêu đề slide
%       \end{tcolorbox}

% }
\newlength{\customwidth}

% \setbeamertemplate{frametitle}{
%     % Xác định chiều rộng của tiêu đề để chỉ định chiều rộng của tcolorbox
%     \settowidth{\customwidth}{\insertframetitle\hspace{6mm}}

%     \begin{tcolorbox}[
%         colback=orange!10,
%         colframe=white,
%         boxrule=0mm, % Độ dày của viền
%         arc=3mm, % Độ cong của góc
%         auto outer arc,
%         boxsep=1mm,
%         left=-1mm, right=2mm, top=0mm, bottom=1mm,
%         width=\customwidth % Đặt chiều rộng của tcolorbox bằng với chiều rộng của tiêu đề
%     ]
%         \color{orange}\insertframetitle
%     \end{tcolorbox}
% }

\setbeamertemplate{frametitle}{
    \settowidth{\customwidth}{\insertframetitle\hspace{6mm}}
    \begin{tcolorbox}[
            enhanced,
            colback=orange!10,
            colframe=white,
            boxrule=0mm,
            arc=3mm,
            auto outer arc,
            boxsep=1mm,
            left=0mm, right=1mm, top=5mm, bottom=1mm,
            width=\customwidth, % Đặt chiều rộng của tcolorbox bằng với chiều rộng giấy
            enlarge left by=-3mm, % Dịch chuyển toàn bộ tcolorbox sang trái 3mm
            overlay={
                    \node[anchor=west] at ([xshift=1mm]frame.west) {\color{orange}\insertframetitle};
                }
        ]
    \end{tcolorbox}
}


